\documentclass[12pt]{article}

\usepackage{amsmath}
\usepackage{amssymb}
\usepackage{amsfonts}
\usepackage{xcolor}
\usepackage{enumerate}
\usepackage{enumitem}
\usepackage{xeCJK}
\usepackage{physics}

%\setCJKmainfont[AutoFakeBold=6,AutoFakeSlant=.4]{WenQuanYi Micro Hei}
\setCJKmainfont[AutoFakeBold=6,AutoFakeSlant=.4]{TW-Kai}

%\setmainfont{URW Gothic L}
%\setmainfont{Century Schoolbook L}
%\setmainfont{Liberation Serif}
\setmainfont{FreeSerif}

\usepackage[a4paper, total={6in, 9in}]{geometry}

\renewcommand{\labelitemi}{$\bullet$}
\renewcommand{\labelitemii}{$\circ$}
%\renewcommand{\labelitemi}{$\blacksquare$}
%\renewcommand{\labelitemii}{$\box$}
%\renewcommand{\labelitemi}{$\blacktriangleright$}
%\renewcommand{\labelitemii}{$\vartriangleright$}

\usepackage{graphicx}
\graphicspath{ {./img/} }

\usepackage{circuitikz}

\usepackage{titlesec}
\usepackage{titletoc}
\usepackage{xCJKnumb}
\usepackage{indentfirst}
\titleformat{\section}{\Large}{\xCJKnumber{\thesection}、}{1em}{}
\titlespacing{\section}{0cm}{0.3cm}{1em}
\renewcommand{\thesubsection}{\arabic{subsection}}
\titleformat{\subsection}{\large}{(\xCJKnumber{\thesubsection})}{1em}{}
\renewcommand{\thesubsubsection}{(\arabic{subsubsection})}
\titleformat{\subsubsection}{}{\thesubsubsection}{0.8em}{}

\usepackage[]{caption2} 
\renewcommand{\figurename}{} 
\renewcommand{\captionlabeldelim}{ }
\renewcommand{\thefigure}{圖(\xCJKnumber{\arabic{figure}}) }

\renewcommand{\theequation}{11.\arabic{equation}}

%\renewcommand{\CJKglue}{\hskip 0.1cm}
%\linespread{1.2}
\setlength{\parskip}{0.5\baselineskip}

\usepackage{tikz}
\usetikzlibrary{shapes.geometric, arrows}
\tikzstyle{process} = [rectangle, minimum width=4in, minimum height=0.5in, text centered, draw=black]
\tikzstyle{text} = [rectangle, minimum width=1in, minimum height=0.5in, text centered, draw=black]
\tikzstyle{arrow} = [thick,->,>=stealth]

\title{
    {\large 國立台灣大學電資學院電機工程學系\\
    普通物理學實驗報告}\\
    \quad \\
    {\large 實驗十一}\\
    滑線電位計
}
\author{}
\date{}
\begin{document}
    \maketitle
    \begin{center}
        \quad\\
        \quad\\

        \quad\\
        \begin{tabular}{rl}
            組別:&第27組\\
            系級:&電機一\\
            學號:&B07901135\\
            &B07901153\\
            姓名:&何國瑋\\
            &張亞薇\\
            指導老師:&賀培銘\text{ }教授\\
            助教:&陳昱嘉\\
            &吳仁凱\\
            實驗日期:&2019/03/05
        \end{tabular}
    \end{center}
    \pagenumbering{gobble}
    \newpage
    \pagenumbering{arabic}

    \section{實驗目的}

        透過一長條狀均勻導體(電阻)製作並校準一「滑線電位計」,
        並利用此電位計,配合一可變電阻,測量一待測電池的電動勢與內電阻。

    \section{實驗原理}

        \subsection{電池的電動勢與內電阻}

            電池在電路中為電荷提供能量,
            而每庫侖電荷在經過電池時所獲得的能量就稱為電動勢。
            同時,由於能量守恆,每庫侖電荷在電路中所消耗的電位能,
            即電位差$V$,與所獲得的電動勢$\varepsilon$相同。
            兩者的單位皆為伏特,即焦耳/庫侖。\\

            \begin{figure}[h]
                \begin{minipage}{0.5\textwidth}
                    \centering
                    
                    \begin{circuitikz}
                        \draw
                        (0,2) to[battery, l=$\varepsilon$, i<=$I$] (3,2)
                        -- (3,0)
                        to[R,l=$R$] (0,0)
                        -- (0,2)
                        ;
                    \end{circuitikz}
                    \caption{理想電池}
                    \label{fig:ideal_battery}
                \end{minipage}
                \begin{minipage}{0.5\textwidth}
                    \centering
                    
                    \begin{circuitikz}
                        \draw
                        \pgfextra{\ctikzset{bipoles/resistor/width=0.4,
                            bipoles/resistor/hieght=0.1}}
                        (-1,2)node[label={above:$A$}]{}--(1,2) 
                        to[R, l=$r$] (2,2)
                        to[battery, l=$\varepsilon$] (3,2)
                        -- (5,2)
                        node[label={above:$B$}]{}-- (5,0)
                        \pgfextra{\ctikzset{bipoles/resistor/width=1,
                            bipoles/resistor/hieght=1}}
                        to[R,l=$R$] (-1,0)
                        -- (-1,2)
                        (0.5,1)--(0.5,3)--(3.5,3)--(3.5,1)--(0.5,1)
                        (0.5,1.5)--(0.25,1.5)--(0.25,2.5)--(0.5,2.5)
                        (-1,2) to[short, i<=$I$](0.5,2)
                        ;
                    \end{circuitikz}
                    \caption{實際電池}
                    \label{fig:real_battery}
                    
                \end{minipage}
            \end{figure}

            在一般的電路中,為了簡化電路,我們都將電池視為上述的理想電池。
            然而,實際上,由於電池中有內電阻($r$)的存在,
            電荷在流經電池,獲得電動勢的同時,也會在內電阻上失去部分的電位能。
            為了方便計算,我們可以將一個真正的電池看作是
            一個電動勢$\varepsilon$的理想電池串連一個內電阻$r$
            (如\ref{fig:real_battery}所示)。
            那麼,根據歐姆定律(Ohm's law),
            只要有任何電流$I$通過,兩端的端電壓$V_{AB}$即為
            \begin{equation}
                \label{eq:v_epsilon_ir}
            V_{AB}=\varepsilon-\Delta V=\varepsilon-Ir<\varepsilon
            \end{equation}
            也就是說,只有在電流$I=0$時$V_{AB}=\varepsilon$,
            隨著通過的電流$I$增加,電池的端電壓$V_{AB}$會越來越小。

        \subsection{滑線電位計的工作原理}

            由於內電阻在電池內部,我們真正能測量的只有電池的端電壓。
            那麼,為了得到真正的電動勢,
            我們勢必得利用$I=0$時$V_{AB}=\varepsilon$這點。
            然而,一般的電壓計在測量電位差時,
            是透過一個檢流計串連一個極大的電阻,
            根據通過的電流$i$以及電阻值$R$來求出兩端的端電壓,
            因此需要從待測電路中汲取部小部分的電流,
            導致我們無法測量到真正的電動勢$\varepsilon$。\\

            \begin{figure}[h]
                \begin{minipage}{0.41\textwidth}
                    \centering

                    \begin{circuitikz}
                        \draw
                        (0,0)node[circ,label={above:$P$}]
                        to[esource](2,0)
                        to[R, l=$R$](3.5,0)
                        --(4,0)node[circ,label={above:$Q$}]

                        (1,-0.45)node[label={above:$G$}]
                        ;
                    \end{circuitikz}
                    \caption{一般電壓計}
                    \label{fig:volmeter}
                \end{minipage}
                \begin{minipage}{0.58\textwidth}
                    \centering

                    \begin{circuitikz}
                        \draw
                        (0,0)
                        to[vR,l_=$R$](2,0)
                        to[battery,l_=$W$](4,0)
                        --(6,0)
                        --(6,-2.5)
                        --(5,-2.5)node[label={above:$C$}]
                        to[R](3.5,-2.5)node[label={above:$B$}]
                        to[R](2,-2.5)
                        to[R](0.5,-2.5)node[circ,label={above:$A$}]
                        --(0,-2.5)
                        --(0,0)

                        (0.5,-2.5)
                        --(0.5, -5)node[circ,label={right:$P$}]

                        (3.5,-5)node[circ,label={right:$Q$}]
                        to[esource](3.5,-3)node[vcc]

                        (3.5,-4.45)node[label={above:$G$}]
                        ;
                    \end{circuitikz}
                    \caption{滑線電位計}
                    \label{fig:slide_volmeter}
                \end{minipage}
            \end{figure}
            
            滑線電位計正好能解決這個問題。
            如\ref{fig:slide_volmeter}所示,
            $AC$之間為一均勻電阻線,$B$為一可移動端點,
            $PQ$為待測電路的兩端點,
            電壓源$W$與可變電阻$R_0$則能提供$AC$端固定的電壓。
            由於$AC$之間為均勻電阻線,
            我們知道
            \begin{equation}
                \label{eq:v_leanier}
            V_{AB}=V_{AC}\times \frac{\overline{AB}}{\overline{AC}}
            \end{equation}
            那由於$A$點和$P$點的電位相同,
            當$V_{PQ}=V_{AB}$時,
            $B$點和$Q$點的電位也會相同,$BQ$之間不會有電流。
            換句話說,將$PQ$接在電池兩端時,
            只要調整$B$點的位置,
            使$BQ$之間的檢流計讀數為零,
            我們便能測得$V_{PQ}=V_{AB}$,
            同時也能確定沒有從電池中汲取電流,
            精確測得無電流通過的電池的端電壓(即電動勢)。

        \subsection{測量待測電池的電動勢與內電阻}
        \label{ssec:measure_r}

           以滑線電位計直接測量電池兩端,
           所測得的端電壓即為電池的電動勢$\varepsilon_{x}$。
           (一般電壓計因為會有微小的電流通過,
           由\ref{eq:v_epsilon_ir}式可知,
           測得的值會略低於$\varepsilon_x$)
           若要求電池的內電阻,
           我們可以使用\ref{fig:measure_r}中的電路:

           \begin{figure}[h]
               \begin{minipage}{1\textwidth}
                    \centering
                    
                    \begin{circuitikz}
                        \draw
                        \pgfextra{\ctikzset{bipoles/resistor/width=0.4,
                            bipoles/resistor/hieght=0.1}}
                        (-1,2)--(1,2) 
                        to[R, l=$r$] (2,2)
                        to[battery, l=$\varepsilon_{x}$] (3,2)
                        -- (5,2)
                        -- (5,0)
                        \pgfextra{\ctikzset{bipoles/resistor/width=1,
                            bipoles/resistor/hieght=1}}
                        to[R,l=$R$] (-1,0)
                        -- (-1,2)
                        (0.5,1)--(0.5,3)--(3.5,3)--(3.5,1)--(0.5,1)
                        (0.5,1.5)--(0.25,1.5)--(0.25,2.5)--(0.5,2.5)
                        (-1,2) to[short, i<=$I$](0.5,2)
                        (-1,0)node[circ,label={left:$P$}]{}
                        (5,0)node[circ,label={right:$Q$}]{}
                        ;
                    \end{circuitikz}
                    \caption{測量內電阻的電路}
                    \label{fig:measure_r}
               \end{minipage}
           \end{figure}

           \noindent 其中,$R$為一已知的可變電阻,
           $V_{PQ}$則是我們要測量的電池的端電壓。
           由於
           \begin{equation}
               \label{eq:vpq_mpsilon}
           V_{PQ}=\frac{R}{R+r}\varepsilon_{x}
           \end{equation}
           透過測得的$V_{PQ}$與$\varepsilon_{x}$,以及已知的$R$,
           便能求出內電阻$r$。
           而為了減少誤差的影響,
           較好的作法是以不同的$R$分別測量$V_{PQ}$並分析數據,
           方法如下:
           
            \subsubsection{作$V-I$圖}
                由(\ref{eq:v_epsilon_ir})式可得$V_{PQ}=\varepsilon_{x}-Ir$,
                也就是說,\ref{fig:vi_example}的三$V-I$圖中,
                $y$軸截距即為$\varepsilon_{x}$,
                斜率$-r$則為負的內電阻值。

            \begin{figure}[h]
                \begin{minipage}{0.5\textwidth}
                    \centering
                    \includegraphics[height=2in]{vi_1.png}
                    \caption{$V-I$圖}
                    \label{fig:vi_example}
                \end{minipage}
                \begin{minipage}{0.5\textwidth}
                    \centering
                    \includegraphics[height=2in]{vr_1.png}
                    \caption{$V-R$圖}
                    \label{fig:vr_example}
                \end{minipage}
            \end{figure}

            \subsubsection{作$V-R$圖}

            由歐姆定律$\varepsilon_x=I(r+R)$,可得$I=\varepsilon_x / (r+R)$,
            代入(\ref{eq:v_epsilon_ir})式並化簡,
            可得
            \begin{equation}
                \label{eq:vr}
            V_{PQ}=\varepsilon_x(\frac{R}{r+R})
            \end{equation}

            而當$R=r$時,$V_{PQ}=\varepsilon_{x}/2$,
            \ref{fig:vr_example}上$V_{PQ}=\varepsilon/2$所對應的可變電阻值$R$
            即為內電阻值$r$。
                
            \subsubsection{作$\frac{1}{V}-\frac{1}{R}$圖}
                
            將(\ref{eq:vr})式取倒數之後可得
            \begin{equation}
                \label{eq:1v1r}
            \frac{1}{V_{PQ}}=\frac{r}{\varepsilon_x}\times \frac{1}{R}+\frac{1}{\varepsilon_x}
            \end{equation}

            因此,在$\frac{1}{V}-\frac{1}{R}$圖中,
            最佳直線的$y$截距為$1/\varepsilon_x$,
            最佳直線的斜率則為$r/\varepsilon_x$,
            乘上$\varepsilon_x$即可得到內電阻值$r$。

            \begin{figure}[h]
                \centering
                \includegraphics[height=2.8in]{1v1r_1.png}
                \caption{$\frac{1}{V}-\frac{1}{R}$圖}
                \label{fig:1v1r_ex}
            \end{figure}
                
                
               
    \section{實驗流程}
        
        \subsection{實驗儀器}

            \quad

            \begin{figure}[h]
                
                \begin{minipage}{0.49\textwidth}
                \centering
                \includegraphics[height=1.5in]{ex1_b.png}
                \caption{實驗裝罝示意圖(實驗A)}
                \label{fig:ex1}
                \end{minipage}
                \begin{minipage}{0.49\textwidth}
                \centering
                \includegraphics[height=1.5in]{ex3_b.png}
                \caption{實驗裝罝示意圖(實驗C)}
                \label{fig:ex3}
                \end{minipage}
                
            \end{figure}
            
        \begin{enumerate}[label={\Alph*. }, leftmargin=2\parindent]
                \item 檢流計:確認是否有電流通過。
                \item 標準電池:已知電動勢的電池,用以校正滑線電位計。
                \item 工作電池:滑線電位計中的$W$,即$AC$之間的電壓來源。
                \item 待測電池:電動勢與內電阻均為未知。
                \item 電阻箱:在測量端電壓時作為已知的可變電阻使用。
                \item 電阻線:即滑線電位計中$AC$之間的均勻電阻線。共六段,每段
                    100cm長,總長600cm。
            \end{enumerate}
            
            \begin{figure}[h]
                \begin{minipage}{0.27\textwidth}
                \centering
                \includegraphics[height=1.5in]{g.jpg}
                \caption{檢流計}
                \label{fig:g}
                \end{minipage}
                \begin{minipage}{0.36\textwidth}
                \centering
                \includegraphics[height=1.5in]{standard_battery.jpg}
                \caption{標準電池}
                \label{fig:standard_battery}
                \end{minipage}
                \begin{minipage}{0.36\textwidth}
                \centering
                \includegraphics[height=1.5in]{working_battery.jpg}
                \caption{工作電池}
                \label{fig:working_battery}
                \end{minipage}
            \end{figure}
            \begin{figure}[h]
                \begin{minipage}{0.42\textwidth}
                \centering
                \includegraphics[height=2in]{test_battery.jpg}
                \caption{待測電池}
                \label{fig:test_battery}
                \end{minipage}
                \begin{minipage}{0.57\textwidth}
                \centering
                \includegraphics[height=2in]{changeable_r.jpg}
                \caption{電阻箱}
                \label{fig:changeable_r}
                \end{minipage}
            \end{figure}

        \subsection{實驗步驟}
            
        \subsubsection{校準滑線電位計}
        \label{sssec:ex1}

        \begin{figure}[h]
            \begin{minipage}{0.34\textwidth}
                \centering
                \includegraphics[height=1.7in]{ex1_a.png}
                \caption{電路圖}
                \label{fig:ex1_a}
            \end{minipage}
            \begin{minipage}{0.65\textwidth}
                \centering
                \includegraphics[height=1.7in]{ex1_b.png}
                \caption{實際配置圖(圖中字母標示為器材編號)}
                \label{fig:ex1_b}
            \end{minipage}
        \end{figure}

        \begin{center}
            \begin{tikzpicture}[node distance=0.7in]
            
                \node (n1) [process] {固定B點於距A點150cm處(可視標準電池所標示之電動勢調整)};
                \node (n2) [process, below of=n1] {工作電池通電後調整D點位置,
                    使檢流計讀數為零};
                \node (n3) [process, below of=n2] {此時AB間電位差即為標準電池電動勢
                ,回推並記錄工作電池端電壓};
                \draw [arrow] (n1) -- (n2);
                \draw [arrow] (n2) -- (n3);
            
            \end{tikzpicture}
        \end{center}
            
        \subsubsection{利用滑線電位計測量待測電池的電動勢$\varepsilon_x$}

        \quad

        \begin{figure}[h]
            \begin{minipage}{0.34\textwidth}
                \centering
                \includegraphics[height=2in]{ex2_a.png}
                \caption{電路圖}
                \label{fig:ex2_a}
            \end{minipage}
            \begin{minipage}{0.65\textwidth}
                \centering
                \includegraphics[height=2in]{ex2_b.png}
                \caption{實際配置圖(圖中字母標示為器材編號)}
                \label{fig:ex2_b}
            \end{minipage}
        \end{figure}

        \begin{center}
            \begin{tikzpicture}[node distance=0.7in]
            
                \node (n1) [process] 
                {將實驗\ref{sssec:ex1}中的標準電池換成待測電池};
                \node (n2) [process, below of=n1] 
                {D點固定,調整B點位置使檢流計讀數為零,記錄AB距離};
            
                \draw [arrow] (n1) -- (n2);
            
            \end{tikzpicture}
        \end{center}
        
        \newpage
        \subsubsection{求待測電池內電阻值$r$}

        \quad

        \begin{figure}[h]
            \begin{minipage}{0.36\textwidth}
                \centering
                \includegraphics[height=1.9in]{ex3_a.png}
                \caption{電路圖}
                \label{fig:ex3_a}
            \end{minipage}
            \begin{minipage}{0.63\textwidth}
                \centering
                \includegraphics[height=1.9in]{ex3_b.png}
                \caption{實際配置圖(圖中字母標示為器材編號)}
                \label{fig:ex3_b}
            \end{minipage}
        \end{figure}

        \begin{center}
            \begin{tikzpicture}[node distance=0.7in]
            
                \node (n1) [process] 
                {將待測電池與電阻箱上的電阻並聯,如\ref{fig:ex3_a}所示};
                \node (n2) [process, below of=n1, xshift=-1.2in, yshift=-0.2in] 
                {移動B點,使檢流計讀數為零};
                \node (n3) [process, below of=n2] 
                {記錄AB之間的距離並推算待測電池的端電壓};
                \node (n4) [process, below of=n3] 
                {改變電阻箱上的電阻值,再次測量端電壓};
                \node (n5) [text, right of=n3, xshift=2.4in] 
                {10個電阻值,重複兩次};
                \node (n6) [process, below of=n4, xshift=1.2in, yshift=-0.2in]
                {根據數據作$V-I$圖,$V-R$圖,以及$\frac{1}{V}-\frac{1}{R}$圖};
                \node (n7) [process, below of=n6]
                {以前面實驗原理第(\xCJKnumber{\ref{ssec:measure_r}})部分提過的方式分別求出內電阻};
                
            
                \draw [arrow] (n1) -- (n2);
                \draw [arrow] (n2) -- (n3);
                \draw [arrow] (n3) -- (n4);
                \draw [arrow] (n4) -| (n5);
                \draw [arrow] (n5) |- (n2);
                \draw [arrow] (n4) -- (n6);
                \draw [arrow] (n6) -- (n7);

            
            \end{tikzpicture}  
        \end{center}

    \section{數據整理與分析}
        
            \subsection{校準滑線電位計}

                已知:

                \begin{itemize}
                    \item 標準電池電動勢:$1.5V$
                    \item AB接點距離:$150.00cm$
                \end{itemize}

                調整D點位置至檢流計讀數為零時:

                \begin{itemize}
                    \item CD接點距離:$486.80cm$
                \end{itemize}

                推得工作電池$W$端電壓:
                $$1.5V\times \frac{486.80cm}{150.00cm}=4.8680V$$

                此時每$100.00cm$電位降落$1Volt$,之後D點固定不動。

            \subsection{利用滑線電位計測量待測電池的電動勢$\varepsilon_x$}
                
                將標準電池換成待測電池,直接以滑線電位計測量電動勢:

                \begin{itemize}
                    \item 檢流計讀數為零時,AB接點距離:$141.38cm$
                \end{itemize}

                推得待測電池電動勢$\varepsilon_x$:
                $$\varepsilon_x = 1V\times \frac{141.38cm}{100.00cm}=1.4138V$$
                
            \newpage
            \subsection{求待測電池內電阻值$r$}

                以十個不同的電阻值,做兩次同樣的實驗
                (完整數據表格附在第\xCJKnumber{\ref{sec:data}}部分的原始數據中)。
                作圖分析內電阻值$r$以及數據圖顯示之電動勢$\varepsilon_x$誤差:

                \subsubsection{$V-I$圖}

                \begin{figure}[h]
                    \begin{minipage}{0.99\textwidth}
                        \centering

                        \includegraphics[height=3in]{vi_1.png}
                        \caption{}
                        \label{fig:vi_1}
                    \end{minipage}
                    \begin{minipage}{0.99\textwidth}
                        \centering

                        \includegraphics[height=3in]{vi_2.png}
                        \caption{}
                        \label{fig:vi_2}
                    \end{minipage}
                \end{figure}

                由圖中可以看出,端電壓V與電流值I為一次線性關係。
                考慮$V=\varepsilon_x-Ir$,實驗結果符合理論。
                從兩次實驗數據的趨勢線中分別可以得到:
                $$
                \begin{cases}
                    &V_{1}=1.4135-24.2610I_1, \quad \varepsilon_{x1}=1.4135(Volt),\quad r_1=24.2610(Ohm)\\
                    &V_{2}=1.4070-23.3100I_2, \quad \varepsilon_{x2}=1.4070(Volt),\quad r_2=23.3100(Ohm)\\
                \end{cases}
                $$
                \begin{itemize}
                    \item $\varepsilon_x$平均值:1.4102(Volt)
                    \item $\varepsilon_x$標準差:0.0046(Volt)
                    \item $r$平均值:23.7855(Ohm)
                    \item $r$標準差:0.6725(Ohm)
                \end{itemize}
                若以實驗二測出的待測電池電動勢($\varepsilon_x=1.4138Volt$)為標準,百分誤差為:
                $$\frac{1.4102-1.4138}{1.4138}\approx -0.25\%$$

                \subsubsection{$V-R$圖}
                    
                \begin{figure}[h]
                    \begin{minipage}{0.99\textwidth}
                        \centering

                        \includegraphics[height=3in]{vr_1.png}
                        \caption{}
                        \label{fig:vr_1}
                    \end{minipage}
                \end{figure}
                \newpage
                \begin{figure}[h]
                    \begin{minipage}{0.99\textwidth}
                        \centering

                        \includegraphics[height=3in]{vr_2.jpg}
                        \caption{}
                        \label{fig:vr_2}
                    \end{minipage}
                \end{figure}
                \noindent 由(\ref{eq:vr})式可推算,當$R$趨近無限大時
                $$\lim_{R\rightarrow \infty}V=\lim_{R\rightarrow \infty}\varepsilon_x(\frac{R}{r+R})=\varepsilon_x$$
                因此實驗結果符合理論。從兩次實驗的圖表中可以得到:
                $$
                \begin{cases}
                    &\varepsilon_{x1}\approx 1.4(Volt),\quad r_1\approx 20(Ohm)\\
                    &\varepsilon_{x2}\approx 1.4(Volt),\quad r_2\approx 20(Ohm)\\
                \end{cases}
                $$
                \begin{itemize}
                    \item $\varepsilon_{x}$平均值:1.4(Volt)
                    \item $r$平均值:20(Ohm)
                \end{itemize}
                由於只是大略從圖形估計的值,在此就不作標準差與百分誤差的討論,
                但得出的$\varepsilon_x$跟我們一開始測量出的$\varepsilon_x$相當接近,
                得出的$r$也與我們用其他方式求出的$r$相當接近,所以結果是符合理論值的。
                \newpage

                \subsubsection{$\frac{1}{V}-\frac{1}{R}$圖}

                \begin{figure}[h]
                    \begin{minipage}{0.99\textwidth}
                        \centering

                        \includegraphics[height=3in]{1v1r_1.png}
                        \caption{}
                        \label{fig:1v1r_1}
                    \end{minipage}
                    \begin{minipage}{0.99\textwidth}
                        \centering

                        \includegraphics[height=3in]{1v1r_2.png}
                        \caption{}
                        \label{fig:1v1r_2}
                    \end{minipage}
                \end{figure}

                從實驗數據的圖形可看出,
                端電壓$V$的倒數與外部電阻$R$的倒數成正比,
                且為一次線性關係。
                由前面實驗原理提到過的(\ref{eq:1v1r})式可知,
                實驗結果符合理論。
                而從實驗結果的趨勢線,
                我們可以得到
                $$
                \begin{cases}
                    &\frac{1}{V_1}=16.520 \frac{1}{R_1} + 0.7302,\quad \frac{r_1}{\varepsilon_{x1}}=16.520,\quad \frac{1}{\varepsilon_{x1}}=0.7302\\
                    &\frac{1}{V_2}=16.079 \frac{1}{R_2} + 0.7283,\quad \frac{r_2}{\varepsilon_{x2}}=16.079,\quad \frac{1}{\varepsilon_{x1}}=0.7283
                \end{cases}
                $$
                \noindent 因此
                $$
                \begin{cases}
                    &\varepsilon_{x1}=1.3695(Volt), \quad r_1 = 22.624(Ohm)\\
                    &\varepsilon_{x2}=1.3731(Volt), \quad r_2 = 22.078(Ohm)
                \end{cases}
                $$
                \begin{itemize}
                    \item $\varepsilon_x$平均值:1.3731(Volt)
                    \item $\varepsilon_x$標準差:0.0026(Volt)
                    \item $r$平均值:22.351(Ohm)
                    \item $r$標準差:0.386(Ohm)
                \end{itemize}
                若以實驗二測出的待測電池電動勢($\varepsilon_x=1.4138Volt$)為標準,百分誤差為:
                $$\frac{1.3731-1.4138}{1.4138}\approx -2.8\%$$


    \section{誤差討論}

        \subsection{影響準確度的因素}
            \begin{itemize}
                \item 一開始使用的標準電池的實際電動勢若與標示的實際電動勢有誤差,
                    會導致校正有誤,影響所有後續實驗的數值。舉例而言,
                    若實際電動勢較大,則校正後的滑線電位計測出來的端電壓都會
                    比實際上的端電壓小。
                \item 所有用來連接的電線實際上都有極小的電阻,
                    雖說在滑線電位計中的BQ導線與AP導線上不會造成影響
                    (因為電流為零),但在實驗多組數據求內電阻時,
                    由於這些微小電阻是與我們控制的可變電阻串連,
                    實際上的電池外部電阻會比我們以為的還大
                    (但如果導線品質夠好,這部份的誤差其實小到可以忽略)。
                \item 電阻線的粗細並不可能完全均勻,
                    因此我們在以長度比作為電阻值比例的時候可能會有些微的偏差。
            \end{itemize}
        \subsection{影響精確度的因素}
            \begin{itemize}
                \item 由於B接點透過鱷魚夾連接,­
                    而鱷魚夾的頭本身有寬度且與電阻線
                    之間有兩個以上的接觸點,因此很難精確判斷
                    AB接點的實際距離。
                \item 雖說檢流計十分敏感,但人眼在對齊指針與刻度時還是會有些微的誤差,
                    所以紀錄距離時檢流計的電流並非真的都是零(但由於檢流計的刻度很小,
                    此誤差也非常小)
                \item 由於溫度會影響導體的電阻值,而我們並不能保證實驗過程中溫度都保持一致,
                    人的體溫和通電等因素都會影響溫度,
                    因此外部電阻的實際值並不完全準確。
                \item 其他人為測量時的誤差,如對齊刻度等。
            \end{itemize}
        \subsection{改進建議}
            針對上述提到的誤差因素中,影響較大的因素,比較能改進的有:
            \begin{itemize}
                \item B接點應採用接觸面積較小的器材,如探針等,
                    這樣比較能準確量出AB接點間的距離。
                \item 電阻線底下的長度刻度應該延長到所有電阻線下面,
                    這樣才能更精確地對齊刻度並測量。
                \item 實驗前可先確保所有器材的品質,如導線與檢流計等。
                    若接觸點有生鏽等情形也會影響接觸時的導電情形,
                    應更換器材,盡量在實驗前排除這些因素。
            \end{itemize}
            
            
        
    \section{思考問題}
        \begin{enumerate}
            \item 回答原理中兩個問題
                \begin{enumerate}[(1)]
                    \item 為什麼只有當B的位置恰使得AB間的電壓等於$V_{PQ}$之值的時候,
                        電流計G的讀數才會是0

                        因為電位補償效應,當一個loop中有方向相反、
                        值相同的電動勢,則總電動勢會因此為0,造成電流$I=0$。
                        根據(\ref{eq:v_epsilon_ir})式,
                        即可無視內電阻的問題直接求得$\varepsilon_x$。 
                    \item 若使用滑線電位計直接測電池內兩端的電壓,
                        就直接可知其電動勢ε了,若用一般的電壓表,
                        測得的值則理論上會略低於電動勢ε,為什麼?
                        
                        因為一般電池內會有內電阻$r$,
                        而一般電壓表中會有一個大的電阻$R$與檢流計串連,
                        所以根據(\ref{eq:vr})式,我們測到的端電壓$V$會是
                        $$V=\varepsilon_x(\frac{R}{r+R})$$
                        因為$R/(r+R)<1$,得到的值必比理論值低。 
                \end{enumerate}
            \item 
                滑線電位計本身使用的電池也有內電阻,
                又線路個接點可能有內電阻,
                這些因素會影響實驗結果嗎?標準電池的內電阻呢?
                
                不會影響。因為滑線電路的測量原理為使產生電位補償效應,
                進而使電流等於0,以此來將內電阻或其他電阻所造成的誤差消除。
                而由此可知,無論誤差電阻總量為何,都可以直接忽略,
                假設標準電池內有內電阻亦然。 
            \item 從此實驗中,什麼條件下我們會說電池沒電了?
                
                沒電是指所測端電壓$V$趨近於0,也就是$Ir$趨近於$\varepsilon$
                (由(\ref{eq:v_epsilon_ir})式)。
                可以推斷因內電阻$r$隨時間變得越來越大,
                所以只要有一點電流,
                就會直接使$Ir$趨近於$\varepsilon$,$V$趨近於0。 
            \item 
                如果待測的電池電動勢曾為現今的5倍左右,
                利用這次實驗架構可以如何量測出此待測電阻的電動勢?

                為了使能供成功量測各種變成原來五倍的較大數值,
                我們可以在校準滑線電位計時,
                將AB接點間的距離變成原本的十分之一:

                \begin{center}
                        \begin{tabular}{|c|c|c|}
                        \hline
                        & 原數值與單位 & 更改後數值與單位\\ \hline
                        標準電池($1.5V$)接點間電阻線長 & $150cm$ & $15cm$\\ \hline
                        電位降落$1Volt$所對應的電阻線長 & $100cm$ & $10cm$ \\ \hline
                        \end{tabular}
                \end{center}
        \end{enumerate}
            
    \section{原始數據}
    \label{sec:data}

        \subsection{校準滑線電位計}
            
                \begin{itemize}
                    \item 檢流計讀數為零時,CD接點距離:$486.80cm$
                    \item 推得工作電池$W$端電壓:$4.8680V$
                \end{itemize}

        \subsection{利用滑線電位計測量待測電池的電動勢$\varepsilon_x$}
            
                \begin{itemize}
                    \item 檢流計讀數為零時,AB接點距離:$141.38cm$
                    \item 推得待測電池電動勢$\varepsilon_x$:$1.4138V$
                \end{itemize}

        \subsection{求待測電池內電阻值$r$}

        \begin{figure}[h]
            \begin{minipage}{0.99\textwidth}
                \centering
                \includegraphics[height=1.7in]{table.png}
                \caption{原始數據表格}
                \label{fig:table}
            \end{minipage}
        \end{figure}

            \newpage

                \subsubsection{$V-I$圖}

                \begin{figure}[h]
                    \begin{minipage}{0.49\textwidth}
                        \centering

                        \includegraphics[height=2in]{vi_1.png}
                        \caption{}
                        \label{fig:vi_1}
                    \end{minipage}
                    \begin{minipage}{0.49\textwidth}
                        \centering

                        \includegraphics[height=2in]{vi_2.png}
                        \caption{}
                        \label{fig:vi_2}
                    \end{minipage}
                \end{figure}

                $$
                \begin{cases}
                    &V_{1}=1.4135-24.2610I_1, \quad \varepsilon_{x1}=1.4135(Volt),\quad r_1=24.2610(Ohm)\\
                    &V_{2}=1.4070-23.3100I_2, \quad \varepsilon_{x2}=1.4070(Volt),\quad r_2=23.3100(Ohm)\\
                \end{cases}
                $$
                \begin{itemize}
                    \item $\varepsilon_x$平均值:1.4102(Volt)
                    \item $r$平均值:23.7855(Ohm)
                \end{itemize}

                \subsubsection{$V-R$圖}
                    
                \begin{figure}[h]
                    \begin{minipage}{0.49\textwidth}
                        \centering

                        \includegraphics[height=2in]{vr_1.png}
                        \caption{}
                        \label{fig:vr_1}
                    \end{minipage}
                    \begin{minipage}{0.49\textwidth}
                        \centering

                        \includegraphics[height=2in]{vr_2.jpg}
                        \caption{}
                        \label{fig:vr_2}
                    \end{minipage}
                \end{figure}

                $$
                \begin{cases}
                    &\varepsilon_{x1}\approx 1.4(Volt),\quad r_1\approx 20(Ohm)\\
                    &\varepsilon_{x2}\approx 1.4(Volt),\quad r_2\approx 20(Ohm)\\
                \end{cases}
                $$
                \begin{itemize}
                    \item $\varepsilon_{x}$平均值:1.4(Volt)
                    \item $r$平均值:20(Ohm)
                \end{itemize}

                \subsubsection{$\frac{1}{V}-\frac{1}{R}$圖}

                \begin{figure}[h]
                    \begin{minipage}{0.49\textwidth}
                        \centering

                        \includegraphics[height=2in]{1v1r_1.png}
                        \caption{}
                        \label{fig:1v1r_1}
                    \end{minipage}
                    \begin{minipage}{0.49\textwidth}
                        \centering

                        \includegraphics[height=2in]{1v1r_2.png}
                        \caption{}
                        \label{fig:1v1r_2}
                    \end{minipage}
                \end{figure}

                $$
                \begin{cases}
                    &\varepsilon_{x1}=1.3695(Volt), \quad r_1 = 22.624(Ohm)\\
                    &\varepsilon_{x2}=1.3731(Volt), \quad r_2 = 22.078(Ohm)
                \end{cases}
                $$
                \begin{itemize}
                    \item $\varepsilon_x$平均值:1.3731(Volt)
                    \item $r$平均值:22.351(Ohm)
                \end{itemize}

    \section{補充資料與閱讀心得}
        
        \noindent 資料來源:College Physics, 28.2-28.6
         

        這章節主要所提為電流與電阻,概念都偏基礎,
        像是電流與電子移動方向的關係、其中個人較沒有注意的是其電流密度的值,
        像是電流密度$J=I/A$($A$為面積、$I$為電流),
        而衍伸出來的物質傳導率的概念$\sigma=J/E$。
        這個項目的重要性跟助教在課堂上提及的趨膚效應有關係,
        不再贅述。除趨膚效應,發現亦跟電阻材料的穩定有關,
        由於過多的能量材料會以熱的方式消耗,為避免過熱將電阻融掉,
        我們已得注意電流密度。影響更重者如超導體,
        還會因為產生磁場而自發性地喪失超導性值。
        發現這不失一個重要的數值。
        
        \noindent 在整章中最讓人有印象的是將電阻傳導率和溫度一併討論的部分。\\
        首先最為基礎的是:
        $$\frac{I}{A}=\sigma \frac{V_1-V_2}{L}\quad (\sigma\text{為傳導率,}L\text{為勞倫茲常數})$$
        \noindent 及其衍伸的:
        $$H=KA\frac{t_1-t_2}{L}\quad 
        \text{($H$是熱流、$K$是熱常數、$t_1, t_2$是起始結束的時間。)}$$

        傳導率和溫度的差別在書中提起的並非只有數學上的相關,
        更有著用維德曼–夫蘭茲定理可以解釋的相關性。
        基本上,電子除了作為一種對「使電流產生」及「決定其速度」兩項很有關係的物質,
        他多多少少也作為傳遞熱能的物質,
        因此我們可以發現有些材質-像有些金屬,
        在電流傳導性上和熱能傳導性的大致符合以下公式:
        $$\frac{K}{\sigma}=3(\frac{k}{e})\times 2T$$
        由此可知熱傳導係數和電傳導係數的關係跟溫度相關。
        這些可以推算在實驗其他歐姆相關實驗可能出現的誤差,
        因為由前述回答問題的答案中我們可以知道,在用滑線電位計測量端電壓時,
        由於會呈現無電流狀態所以可以忽視所有沒考慮到的電阻數值。
         
    \section{參考資料}

        \begin{itemize}
            \item 國立台灣大學普通物理實驗(107學年度):實驗十一 滑線電位計
            \item College Physics, 28.2-28.6 
        \end{itemize}
        
    \section{實驗心得}
    
    基本上,最首先有的心得是工具的正確應用,可以為實驗中和實驗後的結報製作省時省力。
    例如,在收集實驗數據以得出想要的數據項目這點中,
    我們發現excel的利用是一個既效率、又可以因為需要在實驗前準備而加深對公式印象的好習慣。
    除了我們組也因此比其他組還要有效率,也更快在實驗開始後了解接下來動作的目的、應得結果。

    儘管是較為簡易的實驗,也有遇到一些預料之外的問題。
    例如測量長度的誤差,使用的方法為用物品延伸電阻上定點到尺上刻度,
    然而發現所選用物品的厚薄其實會影響量測精準度,因此立刻進行更換成更薄物品的動作。
    還有,有時使儀器歸0太過困難,後來才發現並非我們操作的問題,而是儀器本身的問題,
    因此及時向鄰桌借用儀器才順利完成實驗。及時發現需要修正的點是很重要的事情。

    最後,這個實驗有諸多誤差,例如來自於檢流計是否精確地歸0,
    以及量測電阻線長度的困難。
    個人認為,有時是實驗法的問題,
    但其實有時是有關個人對實驗精準度的追求,
    以及是否僅是要得出一個大致結論的態度。
    在首次進行較高中更為要求精細度的大學實驗時,
    深刻地感受到哪怕是僅是驗證所學的實驗,
    都應該抱持在進行為求假設真實與否的實驗時,應該抱有的嚴謹。 
        
\end{document}
